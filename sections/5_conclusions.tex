% === [ Conclusions ] ==========================================================

\begin{frame}
	\frametitle{Conclusions}

	\begin{enumerate}
		\item Significance of papers
		\item Relation to other research
	\end{enumerate}
\end{frame}

% --- [ Significance of papers ] -----------------------------------------------

\begin{frame}
	\frametitle{Significance of papers}

	Rosenfeld was among the first to formalized fuzzy graph theory in the seminal paper ``Fuzzy graphs'' (1975), which has since received 910 citations (and is also cited in ``Fuzzy morphisms between graphs'').

	\vspace*{2em}

	Perchant and Bloch established a formalism for inexact graph matchings in ``Fuzzy morphisms between graphs''. The paper remains less influential with 61 citations.
\end{frame}

% --- [ Relation to other research ] -------------------------------------------

% how the research relates to the larger picture, what is your personal
% view of the significance of the research. related research, both prior
% and later.

\begin{frame}
	\frametitle{Relation to other research}

% 2019-
%
% TODO: Summary of what happens in the academic community, main reserach happening.

	\begin{block}{Prior-art}
		Rosenfeld was a leading researcher of image analysis and essentially established the field; among others writing the first textbook on the topic (``Picture Processing by Computer'' in 1969, later followed by ``Digital picture processing'' in 1976, a book with 8547 citations).

		\vspace*{2em}

		Similarly, Perchant and Bloch have a background in image analysis.
	\end{block}

	\begin{block}{Follow-up research}
		Bloch went on to applied research related to cognitive science and have among others co-authored the paper ``Diffeomorphic demons: Efficient non-parametric image registration'', which has received 1106 citations to date.

		\vspace*{2em}

		In general, follow-up research seem to be more oriented towards application rather than theory.
	\end{block}
\end{frame}
