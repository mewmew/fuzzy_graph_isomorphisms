% === [ Conclusions ] ==========================================================

\begin{frame}
	\frametitle{Conclusions}

	\todo{skip subdisposition?}

	\begin{enumerate}
		\item Significance of research
		\item Prior-art and follow-up research
	\end{enumerate}

\end{frame}

% === [ Relation to other research ] =============================================

\begin{frame}
	\frametitle{Relation to other research}

	\begin{block}{Significance of research}
		\todo{add note on the significance of the research}

		\begin{itemize}
			\item \todo{add approx quotations (and see if any of the follow-up research is worth reading)}
		\end{itemize}
	\end{block}

% 2019-
%
% TODO: Summary of what happens in the academic community, main reserach happening.

	\begin{block}{Prior-art and follow-up research}
		\begin{itemize}
			\item \todo{add prior art}
			\item \todo{add follow-up research}
		\end{itemize}
	\end{block}

\end{frame}

% === [ Prior-art and follow-up research ] =====================================

% Prior and follow-up research:
%
% Fuzzy mathematical morphologies: a comparative study
% I Bloch, H Maître
% Pattern recognition 28 (9), 1341-1387
% 	citations 468	1995
%
% Fuzzy techniques in image processing
% EE Kerre, M Nachtegael
% Physica
% 	citations 310	2013

% BOOK: Digital picture processing
% Rosenfeld, A. (1976). Digital picture processing. Academic press.
% 8547 citations

% Rosenfeld, A. (1971). Fuzzy groups. Journal of mathematical analysis and applications, 35(3), 512-517.
%
% citations 2465

% Application of theory..
%
% Perchant, A., & Bloch, I. (2002). Fuzzy morphisms between graphs. Fuzzy Sets and Systems, 128(2), 149-168.
%
% 61 citations as of 2019-12-10

% Heavy work in image analysis
%
% Vercauteren, T., Pennec, X., Perchant, A., & Ayache, N. (2009). Diffeomorphic demons: Efficient non-parametric image registration. NeuroImage, 45(1), S61-S72.
%
% 1106 citations as of 2019-12-10

% From wiki: Rosenfeld was a leading researcher in the field of computer image analysis.

% Foundational! -->
%
% Rosenfeld, A. (1975). Fuzzy graphs. In Fuzzy sets and their applications to cognitive and decision processes (pp. 77-95). Academic Press.
%
% 910 citations as of 2019-12-10

% how the research relates to the larger picture, what is your personal
% view of the significance of the research. related research, both prior
% and later.
